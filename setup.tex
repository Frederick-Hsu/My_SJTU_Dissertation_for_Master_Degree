% !TEX root = ./main.tex

\sjtusetup{
  %
  %******************************
  % 注意:
  %   1. 配置里面不要出现空行
  %   2. 不需要的配置信息可以删除
  %******************************
  %
  % 信息录入
  %
  info = {%
    %
    % 标题
    %
    zh / title           = {肺部CT平扫图像中的支气管分割技术研究},
    en / title           = {Research on the technique of pulmonary bronchus segmentation in the CT-scanning images},
    %
    % 标题页标题
    %   可使用“\\”命令手动控制换行
    %
    % zh / display-title   = {上海交通大学学位论文\\ \LaTeX{} 模板示例文档},
    % en / display-title   = {A Sample Document \\ for \LaTeX-based SJTU Thesis Template},
    %
    % 关键词
    %
    zh / keywords        = {支气管气道树分割, 3D-UNet基准网络, 注意力蒸馏, 通道级特征再学习, 三维卷积层},
    en / keywords        = {Bronchial airway segmentation, 3D-Unet baseline network, Attention distillation, Channel-wise feature re-learning, 3D convolutional layer},
    %
    % 姓名
    %
    zh / author          = {徐 \quad{} 赞},
    en / author          = {Xu \quad{} Zan},
    %
    % 指导教师
    %
    zh / supervisor      = {卢宏涛},
    en / supervisor      = {Lu Hong Tao},
    %
    % 副指导教师
    %
    assoc-supervisor  = {刘维平},
    assoc-supervisor* = {Liu Wei Ping},
    %
    % 学号
    %
    id              = {116033210067},
    %
    % 学位
    %   本科生不需要填写
    %
    zh / degree          = {工程硕士},
    en / degree          = {Master of Engineering},
    %
    % 专业
    %
    zh / major           = {计算机科学与工程},
    en / major           = {Computer Science and Engineering},
    %
    % 所属院系
    %
    zh / department      = {电子信息与电气工程学院},
    en / department      = {School of Electronic Information and Electric Engineering},
    %
    % 答辩日期
    %   使用 ISO 格式 (yyyy-mm-dd);默认为当前时间
    %
    % date            = {2014-12-17},
    %
    % 资助基金
    %
    % zh / fund  = {
    %                {国家 973 项目 (No. 2025CB000000)},
    %                {国家自然科学基金 (No. 81120250000)},
    %              },
    % en / fund  = {
    %                {National Basic Research Program of China (Grant No. 2025CB000000)},
    %                {National Natural Science Foundation of China (Grant No. 81120250000)},
    %              },
  },
  %
  % 风格设置
  %
  style = {%
    %
    % 论文标题页 logo 颜色 (red/blue/black)
    %
    % title-logo-color = black,
  },
  %
  % 名称设置
  %
  name = {
    % bib             = {References},
    % ack             = {谢\hspace{\ccwd}辞},
    % achv            = {攻读学位期间完成的论文},
  },
}

% 使用 BibLaTeX 处理参考文献
%   biblatex-gb7714-2015 常用选项
%     gbnamefmt=lowercase     姓名大小写由输入信息确定
%     gbpub=false             禁用出版信息缺失处理
\usepackage[backend=biber,style=gb7714-2015, gbnamefmt=lowercase]{biblatex}
% 文献表字体
% \renewcommand{\bibfont}{\zihao{-5}}
% 文献表条目间的间距
\setlength{\bibitemsep}{0pt}
% 导入参考文献数据库
\addbibresource{bibdata/thesis.bib}

% 脚注格式
\usepackage[perpage,bottom,hang]{footmisc}

% 定义图片文件目录与扩展名
\graphicspath{{figures/}}
\DeclareGraphicsExtensions{.pdf,.eps,.png,.jpg,.jpeg}

% 确定浮动对象的位置,可以使用 [H],强制将浮动对象放到这里(可能效果很差)
% \usepackage{float}

% 固定宽度的表格
% \usepackage{tabularx}

% 使用三线表:toprule,midrule,bottomrule。
\usepackage{booktabs}

% 表格中支持跨行
\usepackage{multirow}
\usepackage{makecell}
\usepackage{ulem}

% 表格中数字按小数点对齐
\usepackage{dcolumn}
\newcolumntype{d}[1]{D{.}{.}{#1}}

% 使用长表格
\usepackage{longtable}

% 附带脚注的表格
\usepackage{threeparttable}

% 附带脚注的长表格
\usepackage{threeparttablex}

% 算法环境宏包
\usepackage[ruled,vlined,linesnumbered]{algorithm2e}
% \usepackage{algorithm, algorithmicx, algpseudocode}

% 代码环境宏包
\usepackage{listings}
\lstnewenvironment{codeblock}[1][]%
  {\lstset{style=lstStyleCode,#1}}{}

% 物理科学和技术中使用的数学符号,定义了 \qty 命令,与 siunitx 3.0 有冲突
% \usepackage{physics}

% 直立体数学符号
\providecommand{\dd}{\mathop{}\!\mathrm{d}}
\providecommand{\ee}{\mathrm{e}}
\providecommand{\ii}{\mathrm{i}}
\providecommand{\jj}{\mathrm{j}}

% 数学符号与工具
\usepackage{bbm}
\usepackage{mathtools}

% 国际单位制宏包
\usepackage{siunitx}[=v2]

% 定理环境宏包
\usepackage{ntheorem}
% \usepackage{amsthm}

% 绘图宏包
\usepackage{tikz}
\usetikzlibrary{shapes.geometric, arrows}

% 一些文档中用到的 logo
\usepackage{hologo}
\providecommand{\XeTeX}{\hologo{XeTeX}}
\providecommand{\BibLaTeX}{\textsc{Bib}\LaTeX}

% 借用 ltxdoc 里面的几个命令方便写文档
\DeclareRobustCommand\cs[1]{\texttt{\char`\\#1}}
\providecommand\pkg[1]{{\sffamily#1}}

% 自定义命令

% E-mail
\newcommand{\email}[1]{\href{mailto:#1}{\texttt{#1}}}

% hyperref 宏包在最后调用
\usepackage[
  bookmarks=true, 
  bookmarksopen=true, 
  colorlinks=true, 
  linkcolor=black, 
  urlcolor=black,
  citecolor=black,
  bookmarksnumbered=true,
  pdfauthor={徐赞}
]{hyperref}

% 自动引用题注更正为中文
\def\equationautorefname{式}
\def\footnoteautorefname{脚注}
\def\itemautorefname{项}
\def\figureautorefname{图}
\def\tableautorefname{表}
\def\partautorefname{篇}
\def\appendixautorefname{附录}
\def\chapterautorefname{章}
\def\sectionautorefname{节}
\def\subsectionautorefname{小节}
\def\subsubsectionautorefname{小节}
\def\paragraphautorefname{段落}
\def\subparagraphautorefname{子段落}
\def\FancyVerbLineautorefname{行}
\def\theoremautorefname{定理}
