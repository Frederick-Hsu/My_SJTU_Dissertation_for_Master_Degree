% !TEX root = ../main.tex

\chapter{总结与展望}\label{chap:summary_and_outlook}

\section{全文工作总结}

本文研究肺部支气管气道树的分割技术,对于CT扫描图像数据,采用3D-UNet卷积网络进行监督学习的图像分割。本文首先介绍研究的背景和意义,肺部支气管气道树分割的研究现状和发展趋势,
指出支气管气道树分割问题目前面临的两大挑战:1)支气管体素在CT图像中的比例太低,造成了严重的类别不平衡;2)全局尺寸和局部尺寸的特征信息差异悬殊,因支气管的树状结构决定了枝干
支气管的粗糙特征和末梢支气管的精细特征同时存在,卷积层和池化层难以同时学习到这两种差异巨大的特征。针对这些挑战,本文提出2种方法,即注意力蒸馏方法和通道级特征再学习方法,来解决
这些问题,它们也是本文的两大创新点。

本文围绕以下三点展开研究:
\begin{enumerate}
    \item {\heiti 以3D-UNet网络作为基准分割模型}

    我们为3D-UNet基准网络设计了9个卷积层块,下采样路径每个卷积层块含有2个Conv3d卷积层、2个InstanceNorm3d归一化层和2个ReLU激活函数,相邻卷积层块用最大池化层MaxPool3d
    相连(见图\ref{fig:convblock})。上采样路径每个反卷积层块含有2个Conv3d反卷积层,相邻反卷积层块则用Upsample层相连(见图\ref{fig:deconvblock})。3D-UNet网络(结构
    如图\ref{fig:3DUNetStructure})从ATM22数据集读入切割成$D80 \times H192 \times W304$的小立方体数据,使用Dice Loss损失函数和Focal Loss损失函数计算预测值与真实值
    之差。我们在上海交通大学高性能计算中心的AI超算上进行训练,训练了60个迭代周期。我们从假阳性率FPR、假阴性率FNR、灵明度Sensitivity、精度Precision、DSC、检测到的分支BD和j检测
    到的树长TLD共7个指标来评价分割的性能和效果。

    \item {\heiti 提出注意力蒸馏方法改进基准模型}

    3D-UNet基准网络对支气管气道树的分割在假阴性率上表现不佳,我们受注意力转移和知识蒸馏的启发,提出一种注意力蒸馏的方法来提高对支气管气道树特征的聚焦。注意力蒸馏是沿着通道维
    累积$C$个特征立方体,蒸馏浓缩到1个特征立方体(见图\ref{fig:attention_distillation}),起到聚焦于支气管主目标的作用。我们设计实现了注意力蒸馏模块,添加到上采样路径的
    反卷积层块之后,以增加一个额外的梯度。如何计算这额外增加的梯度的损失,我们给出了注意力蒸馏的梯度损失算法\ref{algo:ad_loss}。可视化注意力蒸馏后的图片效果
    (见图\ref{fig:ad_effect}和图\ref{fig:ad_effect}),可以观察到注意力的焦点在逐渐转移,从粗大的气管转移到细小的支气管。经过注意力蒸馏方法改进后的3D-UNet网络,
    进行了一次对比实验,实验结果表明抑制住了一些假阴性率,提高了分割模型的性能。

    \item {\heiti 采用通道级特征再学习方法提取精细特征}

    3D-UNet基准网络经过注意力蒸馏方法改进后,还存在一些不足,模型对末梢支气管的分割能力还不够。我们提出了第2种改进方法:通道级特征再学习方法。我们详细阐述例特征再学习方法的
    基本原理,计算方法与过程。特征再学习方法的重点在于对特征立方体$D \times H \times W$分别沿着Depth、Height和Width方向进行积分,挤压成$D \times 1 \times 1$, 
    $1 \times H \times 1$和$1 \times 1 \times W$的形状(见图\ref{fig:spatial_integration})。这种挤压实际上是一种两次累积求和的过程,它借鉴了注意力蒸馏的思想。
    我们还为特征再学习方法设计了2个卷积操作,为了是将多个通道进行重组,让有信息性的通道得到加强,而无信息性的通道就被抑制削弱。通道级特征再学习方法起的作用是在训练过程中,使重要
    的气道(粗糙的枝干支气管和精细的末梢支气管)增加权重逐渐被重视,而无关区域则降低权重而慢慢被忽略。

    我们对3D-UNet + AD网络再次改进,加入通道级特征再学习功能模块(见图\ref{fig:3dunet_ad_fr}),并进行一次综合实验。综合实验结果(见表\ref{fig:airway_segmentation_overview})
    表明气道树三维模型在假阴性率和假阳性率都得到了抑制,分割得到了完整清晰的气道树。完成了综合实验后,我们跟经典的网络方法对比了FPR、DSC、BD和TLD性能指标。我们的方法在DSC和
    TLD两项指标上领先,但跟SOTA的方法相比则在FPR、DSC和BD指标上落后了。
\end{enumerate}

\section{研究展望}

本文针对肺部支气管气道树的分割以3D-UNet网络为基准模型,提出了2项改进方法:注意力蒸馏方法和通道级特征再学习方法。这些改进提高了分割模型的性能,但相比于SOTA的模型,我们落后了。
因此本研究的第一个展望就是:

{\heiti A. 引入新的机制,继续改进,力争超过SOTA的方法。} 

支气管是一种树状结构,以每一个分支作为分割对象,可以设计成一种树网络模型。每扩展出一个分支,相应地树的深度增加1,分支与分支之间的连接就构成了树的一条边。以这样的树网络模型对
支气管气道树进行分割。 另外受NLP的注意力机制对Transformer巨大成功的鼓舞,我们在后面将引入Transformer来分割医疗图像,其将会大幅提高支气管气道树的分割性能与效果。我们要继续改进,
力争要超过SOTA的方法。

{\heiti B. 应用本研究成果帮助开发导管手术机器人}

本文在\ref{sec:application_prospect}小节就提到本研究的应用前景,我们要开发支气管导管手术机器人。本研究的一个直接动因就是为了开发导管手术机器人, 分割出来的精密的气道树
三维模型可提供给导管手术机器人,引导手术器械在患者肺部支气管里导航至目标的病灶部位。我们的研究使命是治疗疾病,帮助患者恢复健康。

{\heiti C. 扩展到更多的管状对象的分割}

与支气管这样的管状器官或组织类似,人体内还有很多,诸如肺部动脉/静脉血管,心脏的冠脉血管等。我们拟将本文中的3D-UNet + AD + FRL网络模型扩展到肺部血管的分割。后面还会将其扩展
到心脏血管的分割中去,这会给心血管疾病的诊断和治疗带来很大的帮助,其中冠脉支架的植入就需要精确的血管分割,帮助定位血管栓塞和硬化的位置。

{\heiti D. 引入弱监督学习和迁移学习,解决沉重的标注负担}

医疗图像的标注是一项耗时、费力枯燥且成本高昂的工作,目前大量的医疗图像非常缺乏高质量的标注。如何解决沉重的标注负担,我们将研究引入弱监督学习和迁移学习,配合少量的标注来完成
较大批量的自动分割任务。这将大大降低医疗负担,帮助临床医生更好地诊断和治疗。