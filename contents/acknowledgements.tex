% !TEX root = ../main.tex

\begin{acknowledgements}
  本论文经历了一个曲折的过程,历经千辛万苦走到今天终于完成了。

  现在回过头来回顾过去,心中有很多感慨,有许多感激之言。首先,我要感谢卢宏涛教授、刘维平副教授两位导师对我的论文指导,也要感谢我的上任导师朱弘恣教授(我的毕业论文经历
  过一次换课题换导师)。

  我要特别感谢上海交通大学高性能计算中心的费晓舒老师、郭武老师,感谢他们免费给我提供宝贵的计算资源(给我分配了8张NVIDIA Tesla V100显卡 512GB HBM2显存,2颗Intel Xeon铂金8168 CPU, 
  1.5TB DDR4共享内存,30TB NVMe SSD共享存储)和全程的技术支持。本论文是在上海交通大学高性能计算中心的$\pi 2.0$ AI超算上完成计算任务的。
  若没有费晓舒、郭武两位老师的大力支持,我根本无法完成此论文。

  我还要感谢秦玉磊学长、谢昭智博士的指导和帮助,感谢吴娜学姐的督促和支持帮助。我要感谢前同事丁小柳先生、李志坚先生的帮助和技术指导。

  感谢张忠能老师、姚天昉老师的支持帮助。

  我要感谢我的父母和弟弟徐攀,非常感谢你们的支持和持续的监督。感谢余学琴姐姐的支持。

  我要感谢自从考进上海交通大学硕士研究生以来帮助和支持过我学业的老师和同学。

  另外,本论文的撰写使用了开源的\href{https://github.com/sjtug/SJTUThesis}{\sjtuthesis}上海交通大学学位论文\LaTeX{}排版模板, 
  感谢 \href{https://github.com/weijianwen}{@weijianwen} 学长一直以来的开发和维护工作。
  感谢 \LaTeX{} 和 \href{https://github.com/sjtug/SJTUThesis}{\sjtuthesis},帮我自动排版,节省了大量时间。

  最后衷心表示我的感谢!

  % 感谢那位最先制作出博士学位论文 \LaTeX 模板的交大物理系同学!
  % 感谢 William Wang 同学对模板移植做出的巨大贡献!
  % 感谢 \href{https://github.com/weijianwen}{@weijianwen} 学长一直以来的开发和维
  % 护工作!
  % 感谢 \href{https://github.com/sjtug}{@sjtug} 以及
  %  \href{https://github.com/dyweb}{@dyweb} 对 0.9.5 之后版本的开发和维护工作!
  % 感谢所有为模板贡献过代码的同学们, 以及所有测试和使用模板的各位同学!
  % 感谢 \LaTeX 和 \href{https://github.com/sjtug/SJTUThesis}{\sjtuthesis},帮我节
  % 省了不少时间。

\end{acknowledgements}
