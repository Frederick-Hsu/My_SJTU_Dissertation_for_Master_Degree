% !TEX root = ../main.tex

\chapter{复现实验结果}\label{chap:reproduce_experiment}

为复现注意力蒸馏后的效果,请查看以下一些代码片段:
\begin{codeblock}[language=python]

import numpy as np
import SimpleITK as sitk
from matplotlib import pyplot as plt

def load_CT_scan_3D_image(niigz_file_name):
    itkimage = sitk.ReadImage(niigz_file_name)
    numpyImages = sitk.GetArrayFromImage(itkimage)
    numpyOrigin = np.array(list(reversed(itkimage.GetOrigin())))
    numpySpacing = np.array(list(reversed(itkimage.GetSpacing())))
    return numpyImages, numpyOrigin, numpySpacing

#================================================================================

ATM_074_0000_niigz_files = ["ATM_074_0000-feat6.nii.gz", 
                            "ATM_074_0000-feat7.nii.gz", 
                            "ATM_074_0000-feat8.nii.gz", 
                            "ATM_074_0000-feat9.nii.gz"]

feat6_image_npy, feat6_origin, feat6_spacing = 
	load_CT_scan_3D_image(ATM_074_0000_niigz_files[0])

feat7_image_npy, feat7_origin, feat7_spacing = 
	load_CT_scan_3D_image(ATM_074_0000_niigz_files[1])

feat8_image_npy, feat8_origin, feat8_spacing = 
	load_CT_scan_3D_image(ATM_074_0000_niigz_files[2])

feat9_image_npy, feat9_origin, feat9_spacing = 
	load_CT_scan_3D_image(ATM_074_0000_niigz_files[3])
	

depth, height, width = feat6_image_npy.shape

plt.figure(figsize=(10, 10))
plt.imshow(np.flipud(feat6_image_npy[:, height//2, :]))
plt.imshow(np.flipud(feat6_image_npy[:, height//2, :]), cmap="gray")

plt.figure(figsize=(10, 10))
plt.imshow(np.flipud(feat7_image_npy[:, height//2, :]))
plt.imshow(np.flipud(feat7_image_npy[:, height//2, :]), cmap="gray")

plt.figure(figsize=(10, 10))
plt.imshow(np.flipud(feat8_image_npy[:, height//2, :]))
plt.imshow(np.flipud(feat8_image_npy[:, height//2, :]), cmap="gray")

plt.figure(figsize=(10, 10))
plt.imshow(np.flipud(feat9_image_npy[:, height//2, :]))
plt.imshow(np.flipud(feat9_image_npy[:, height//2, :]), cmap="gray")
\end{codeblock}

\chapter{Maxwell Equations}

选择二维情况,有如下的偏振矢量:
\begin{subequations}
  \begin{align}
    {\bf E} &= E_z(r, \theta) \hat{\bf z}, \\
    {\bf H} &= H_r(r, \theta) \hat{\bf r} + H_\theta(r, \theta) \hat{\bm\theta}.
  \end{align}
\end{subequations}
对上式求旋度:
\begin{subequations}
  \begin{align}
    \nabla \times {\bf E} &= \frac{1}{r} \frac{\partial E_z}{\partial\theta}
      \hat{\bf r} - \frac{\partial E_z}{\partial r} \hat{\bm\theta}, \\
    \nabla \times {\bf H} &= \left[\frac{1}{r} \frac{\partial}{\partial r}
      (r H_\theta) - \frac{1}{r} \frac{\partial H_r}{\partial\theta} \right]
      \hat{\bf z}.
  \end{align}
\end{subequations}
因为在柱坐标系下,$\overline{\overline\mu}$ 是对角的,所以 Maxwell 方程组中电场
$\bf E$ 的旋度:
\begin{subequations}
  \begin{align}
    & \nabla \times {\bf E} = \ii \omega {\bf B}, \\
    & \frac{1}{r} \frac{\partial E_z}{\partial\theta} \hat{\bf r} -
      \frac{\partial E_z}{\partial r}\hat{\bm\theta} = \ii \omega \mu_r H_r
      \hat{\bf r} + \ii \omega \mu_\theta H_\theta \hat{\bm\theta}.
  \end{align}
\end{subequations}
所以 $\bf H$ 的各个分量可以写为:
\begin{subequations}
  \begin{align}
    H_r &= \frac{1}{\ii \omega \mu_r} \frac{1}{r}
      \frac{\partial E_z}{\partial\theta}, \\
    H_\theta &= -\frac{1}{\ii \omega \mu_\theta}
      \frac{\partial E_z}{\partial r}.
  \end{align}
\end{subequations}
同样地,在柱坐标系下,$\overline{\overline\epsilon}$ 是对角的,所以 Maxwell 方程
组中磁场 $\bf H$ 的旋度:
\begin{subequations}
  \begin{align}
    & \nabla \times {\bf H} = -\ii \omega {\bf D}, \\
    & \left[\frac{1}{r} \frac{\partial}{\partial r}(r H_\theta) - \frac{1}{r}
      \frac{\partial H_r}{\partial\theta} \right] \hat{\bf z} = -\ii \omega
      {\overline{\overline\epsilon}} {\bf E} = -\ii \omega \epsilon_z E_z
      \hat{\bf z}, \\
    & \frac{1}{r} \frac{\partial}{\partial r}(r H_\theta) - \frac{1}{r}
      \frac{\partial H_r}{\partial\theta} = -\ii \omega \epsilon_z E_z.
  \end{align}
\end{subequations}
由此我们可以得到关于 $E_z$ 的波函数方程:
\begin{equation}
  \frac{1}{\mu_\theta \epsilon_z} \frac{1}{r} \frac{\partial}{\partial r}
  \left(r \frac{\partial E_z}{\partial r} \right) + \frac{1}{\mu_r \epsilon_z}
  \frac{1}{r^2} \frac{\partial^2E_z}{\partial\theta^2} +\omega^2 E_z = 0.
\end{equation}
