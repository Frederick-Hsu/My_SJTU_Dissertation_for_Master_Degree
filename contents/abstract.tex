% !TEX root = ../main.tex

\begin{abstract}
检视医疗图像是现代医疗诊断中普遍使用的诊疗手段,通过计算机断层扫描患者的身体部位,
获得病变部位或器官组织的切片图像。医生检视这些CT扫描图像,辨识、诊断出病灶部位。
本文的研究对象就是病人肺部的CT扫描图像,目标是分割提取出精确的支气管气道树的三维模型。
支气管气道树是很多肺病诊疗的重要参考,目前基于卷积神经网络的图像分割技术在气道树分割提取
方面面临严重类别不平衡等挑战。本文提出2种新方法,注意力蒸馏方法和通道级特征再学习方法,来解决
这些问题挑战。本文构建以3D-UNet网络为基础的基准模型,采用ATM22数据集在AI超算平台上训练分割模型。

3D-UNet网络监督学习对于极度稀疏标注的支气管体数据存在假阳性误检和假阴性漏检的不足,受注意力
迁移和知识蒸馏启发而来的注意力蒸馏方法帮助模型聚焦于支气管树状结构特征,摒弃大量无用背景的干扰,
提取出气道树枝干和细小分支。在3D-UNet上采样路径增加注意力蒸馏模块进行改进,实验证明注意力蒸馏方法
可有效降低假阳性率和假阴性率。

气道树中同时存在枝干支气管体素的粗糙特征和末梢支气管体素的精细特征,精细特征会因池化层降低分辨率而逐渐被“擦除掉”。
不同空间位置有不同的重要性,通道级特征再学习方法通过把卷积层输出的特征通道重新组合,在训练过程中使重要的气道
特征增加权重而被重视,使无关的区域降低权重而被忽略。在3D-UNet每个卷积层和池化层之间插入特征再学习模块进行改进。
综合实验结果表明末梢支气管分割良好,性能指标得到明显提升。

  摘要页的下方注明本文的关键词(4~6个)。
\end{abstract}

\begin{abstract*}
  Examing the medical images is a common method used in modern medical diagnosis. By scanning the patient's body through CT, 
  clinicians obtain and examine the slices of lesion part or organ tissue, then make the disease diagnosis. In this paper we
  investigate the pulmonary CT scan images, for the goal of extracting the accurate airway tree 3D model. Bronchial airway
  tree is an important reference for the therapy of many lung diseases. Currently CNN-based airway tree segmentation is facing 
  some challenges like severe class imbalance. We have proposed 2 new methods in this paper to trackle the problem: attention distillation (AD) method 
  and channel-wise feature re-learning (FRL) method. Based on the 3D-UNet baseline network, we adopt the ATM22 dataset to train a segmentation model on
  the AI supercomputer.

  3D-UNet supervised learning has shortcomings of false positive and false negative issues, when applying it on the extremely sparse annotated volumetric 
  data. Motivated by the attention transfer and knowledge distillation, the attention distillation method is built to focus on the tree structure of bronchial
  airway. It can discard the interference of massive useless background, extract the main airway branches and some small ones. We add AD modules onto the decoder
  path to improve the 3D-UNet baseline. experiment has shown that attention distillation method can effectively reduce the FNR and FPR.

  The airway tree has both the coarse features of bronchial voxels and the fine features of distal bronchial voxels, the fine features could be gradually 
  "erased" due to resolution reduction in the pooling layer. Different spatial positions have different degree of importance. FRL method recombines the feature channels which 
  are output from convolutional layer, it has the crucial airway regions preferred gradually with higher weights, while has the uninformative regions negleted with lower weights.
  The FRL module is inserted between each convolutional layer and pooling layer of 3D-UNet, after that a comprehensive experiment showed that the distal bronchi had 
  been well segmented, and the performance metrics was improved obviously.
\end{abstract*}
